\documentclass[a4paper,10pt]{article}
\usepackage[utf8]{inputenc}
\usepackage{markdown}

%opening
\title{Offensive security\\lab-report}
\author{Moritz Rupp}

\begin{document}

\maketitle

\begin{abstract}
\noindent This document contains reports about the laboratory of the 5th semester module offensive security. 
\end{abstract}
\tableofcontents
\newpage
\section{Lab2}
\subsection{Exercise 2.2}
\begin{center}
\textbf{https://www.hs-albsig.de} 
\end{center}
\$\raisebox{-0.9ex}{\~{}} nslookup hs-albsig.de \\
IP: 94.186.153.201\\ 


\noindent\$\raisebox{-0.9ex}{\~{}} nmap -sC -sV -oA 94.186.153.201\\
Open ports: 80(HTTP), 443(HTTPS)\\
Server: nginx 1.16.1\\
HTTP Version: 1.1 
\\
\\
\$\raisebox{-0.9ex}{\~{}} whois hs-albsig\\
SOA: ns.hs-albsig.de. (141.87.109.5)\\  
Namerserver:
\begin{itemize}
\item Nserver: dns1.belwue.de
\item Nserver: dns3.belwue.de
\item Nserver: dns5.belwue.de
\item Nserver: ns.hs-albsig.de 141.87.109.5
\end{itemize}

\noindent\$\raisebox{-0.9ex}{\~{}} fierce --domain hs-albsig --wide\\
Some if the found subdomains, all belonging to the main host of hs-albsig!\\
To see the full list, take a look at the file 'fierceoutput.txt'.  
\\
\begin{itemize}
 \item helpdesk.hs-albsig.de. - 141.87.114.175
 \item info.hs-albsig.de. - 141.87.114.205
 \item konferenz.hs-albsig.de. - 141.87.115.200
 \item intern.hs-albsig.de. - 141.87.109.198
 \item autodiscover.hs-albsig.de. - 141.87.109.198
 \item h1crelay1.hs-albsig.de. -141.87.109.200
 \item intranet.hs-albsig.de. - 141.87.109.198
 \item jobs.hs-albsig.de. - 62.204.161.138
 \item live.hs-albsig.de. - 141.87.190.3
 \item login.hs-albsig.de. - 194.98.248.75
 \item mail.hs-albsig.de. - 141.87.114.190
 \item mailig.hs-albsig.de. - 54.73.30.56
 \item ntp.hs-albsig.de. - 141.87.190.5
 \item pki.hs-albsig.de. - 141.87.109.3
 \item proxy.hs-albsig.de. - 141.87.109.4
 \item datascience.hs-albsig.de. - 141.87.109.223
  

\end{itemize}

\vspace{10mm}
\noindent The same results can be achived via recon-ng and the hackertarget module.\\
\newline
\noindent\$\raisebox{-0.9ex}{\~{}} recon-ng\\
\$\raisebox{-0.9ex}{\~{}} marketplace search hackertarget\\
\$\raisebox{-0.9ex}{\~{}} marketplace install hackertarget\\
\$\raisebox{-0.9ex}{\~{}} modules load hackertarget\\ 
\$\raisebox{-0.9ex}{\~{}} options set SOURCE hs-albsig.de\\
\$\raisebox{-0.9ex}{\~{}} run \dots

\newpage
\subsection{Exercise 2.3}
Out of my 10 email adresses, 3 have been part of a leagage.
A paid licence is required to view the actuall document content.
It is possible tho to see the origin of the leagage.\\
Leagage origins:
\begin{itemize}
 \item Dailymotion.com
 \item Dropbox.comment
 \item Linux-Foren.com
 \item RepZ.eu
 \item hqcombo.top
\end{itemize}
\vspace{8mm}
Phonebook.nz was able to find over 500 email adresses of albstadt uni members. The majority of leagages matched the above listing.
With the help of tools and websites such as lullar.com, mSpy and peekyou i was able to find the according social media accounts of several email adresses. 
\subsection{Exercise 2.4}

\end{document}
